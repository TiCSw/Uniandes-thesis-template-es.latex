% To refer to elements in the text
\newcommand{\ch}[1]{\ref{ch:#1}}
\newcommand{\secc}[1]{\ref{sec:#1}}
\newcommand{\fig}[1]{\ref{fig:#1}}
\newcommand{\lst}[1]{\ref{lst:#1}}


% Basic commands for elements
\newcommand{\element}[1]{\texttt{#1}}
\newcommand{\method}[1]{\texttt{#1}}
\newcommand{\attribute}[1]{\texttt{#1}}
\newcommand{\event}[1]{\texttt{\textbf{#1}}}
\newcommand{\role}[1]{\texttt{\textbf{#1}}}
\newcommand{\state}[1]{\textit{#1}}
\newcommand{\action}[1]{\textit{#1}}
\newcommand{\transition}[1]{\textit{#1}}

% Abbreviations
\newcommand{\el}[1]{\element{#1}}
\newcommand{\ev}[1]{\event{#1}}
\newcommand{\tr}[1]{\transition{#1}}
\newcommand{\es}[1]{\state{#1}}
\newcommand{\met}[1]{\method{#1}}
\newcommand{\att}[1]{\attribute{#1}}
\newcommand{\act}[1]{\action{#1}}

% Commands for the development process
\newcommand{\actP}[1]{#1} % ACTIVITIES
\newcommand{\roleP}[1]{#1} % ROLES
\newcommand{\art}[1]{#1} %ARTIFACTS
\newcommand{\tool}{\texttt} % TOOLS

% For embedded code
\newcommand{\xml}[1]{\texttt{#1}}

%Metamodels and tools
\newcommand{\miniBPMN}{\texttt{MiniBPMN}}
\newcommand{\XTM}{\texttt{XTM}}
\newcommand{\obj}[1]{\textbf{#1}}
\newcommand{\XPM}{\texttt{XPM}}
\newcommand{{\CTF}}{\tool{CTF}}


% DEFINE THE ``ownFigure'' command
\newcommand{\ownFigure}[5]
{
\begin{figure}[#1] %  figure placement: here, top, bottom, or page
   \centering
   \includegraphics[width=#5\columnwidth]{images/#2} 
   \caption{#4}
   \label{fig:#3}
\end{figure}
}

% List of Comentarios
\newcommand{\listComentariosName}{Comentarios en el texto}
\newlistof{comentarios}{comm}{\listComentariosName}


% DEFINE THE ``comments'' ENVIRONMENT
\ifthenelse {\boolean{showComments}}
{
		
	\newenvironment{comments}[1][MS]
	{
		\singlespacing
		\refstepcounter{comentarios}
		\addcontentsline{comm}{comentarios}{ Comentarios en \thesection~-- #1}
		\begin{center}
		\begin{minipage}{0.99\columnwidth}
		\begin{framed}
		\begin{itshape}	[#1]
	}
	{
		\end{itshape} 
		\end{framed}
		\end{minipage}
		\end{center}
	}
}
{
	\excludecomment{comments}
}

% DEFINE THE ``TODO'' COMMAND
\newcommand{\TODO}[2]
{  
	\refstepcounter{todos}
		{\color{red}
		{
			\begin{center}
				\begin{minipage}{0.9\columnwidth}
					\footnotesize
					\textbf{TODO \thetodos. #1:}
					
					\begin{ttfamily}
					\begin{flushleft}
					#2
					\end{flushleft}
					\end{ttfamily}			
				\end{minipage}
			\end{center}
		 }}

	\addcontentsline{todo}{todos}{ \thetodos:~\thesection~-- #1}
}

% List of TODOS
\newcommand{\listTodosName}{Pending TODO}
\newlistof{todos}{todo}{\listTodosName}

\ifthenelse {\boolean{showTodos}}
{}
{
	\renewcommand{\TODO}[2]{}
}

% DEFINE THE ``contribution'' ENVIRONMENT
\newenvironment{contribution}[1]
{
	\begin{center}
	\begin{minipage}{\columnwidth}
	\begin{framed}
	\begin{itemize} \item\textbf{#1}
	
}
{
	\end{itemize}
	\end{framed}
	\end{minipage}
	\end{center}
}


% To display grammars
\newcommand{\ccrule}[2]
{ \begin{minipage}[t]{0.23\columnwidth}#1\end{minipage} &
  ::= & 
 	\begin{minipage}[t]{0.63\columnwidth}
	#2 \end{minipage}  \\ \hline}
	
\newcommand{\token}[1]{\texttt{\color{blue}#1}}
\newcommand{\chars}[1]{\texttt{\textbf{\color{red}#1}}}

\newcommand{\tokendef}[1]
{\begin{minipage}{0.95\columnwidth}\texttt{#1}\end{minipage}}